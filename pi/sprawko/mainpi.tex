\documentclass[a4paper,12pt]{article}
\usepackage{polski}
\usepackage[utf8]{inputenc}
\usepackage{graphicx}
\usepackage{listings}
\usepackage{color}
\usepackage{csvsimple}
\usepackage{pgfplots}

\definecolor{mygreen}{rgb}{0,0.6,0}
\definecolor{mygray}{rgb}{0.5,0.5,0.5}
\definecolor{mymauve}{rgb}{0.58,0,0.82}

\providecommand{\e}[1]{\ensuremath{\times 10^{#1}}}


\lstset{
    language=c,
    basicstyle=\footnotesize,
    numberstyle=\footnotesize,
    numbers=left,
    frame=single,
    commentstyle=\color{mygreen},
    tabsize=1,
    title=\lstname,
    escapeinside={\%*}{*)},
    breaklines=true,
    breakatwhitespace=true,
    framextopmargin=2pt,
    framexbottommargin=2pt,
    inputencoding=utf8,
    extendedchars=true,
    literate={ł}{{\/l}}1 {ń}{{\'n}}1,
}
\title{Obliczanie wartości $\pi$ z wykorzystaniem wzoru Leibniza}
\author{Rafał Selewońko}
\begin{document}
\maketitle
 
% pierwsza sekcja
\section{Zadanie}\label{sec:zadanie}
Napisać program obliczający liczbę Pi do określonej precyzji (należy przetestować na komputerach jaka będzie rozsądna wartość - tak aby program wykonywał się kilka sekund). Następnie należy zrównoleglić go przy pomocy OpenMP (dla różnej liczby wątków). Dodatkowo należy policzyć i sporządzić wykresy przyśpieszenia i wydajności. Do oceny należy przedstawić raport z wykonania zadania zawierający wydruk programu + wykresy.

\pagebreak
\section{Rozwiązanie}\label{sec:kod}

\lstinputlisting[label=pi]{pi.c}
\pagebreak
\lstinputlisting{main.c}

\lstinputlisting{test.sh}

\section{Wyniki badań}

\begin{table}[t]
\centering
\csvautotabular{wyniki.csv}
\caption{Wyniki testu.}
\label{tabela}
\end{table}

Czas wykonania implementacji jednowątkowej (plik pi.c) wyniósł 5,554 sekundy. Implementacja wielowątkowa (plik main.c) uruchomiona z jednym wątkiem wykonuje się 1.13 razy dłużej niż wersja wielowątkowa. Dla liczby wątków większej niż jeden widać bardzo duże przyspieszenie. Testy były wykonywane na maszynie 4 rdzeniowej z hyperthreadingiem dlatego przyspieszenie rośnie do 8 wątków a dla większej ilości utrzymuje się na podobnym poziomie. W tabeli przedstawiono czas wykonania implementacji wielowątkowej dla ilości wątków od 1 do 32.

\begin{figure}[ht!]
    \centering
	\begin{tikzpicture}
		\begin{axis}[xlabel={Ilość wątków}, ylabel={Czas}, legend style={legend pos=outer north east}]
		\addplot table [x=Wątki, y=Czas, col sep=comma] {wyniki.csv};
		\end{axis}
	\end{tikzpicture}
    \caption{Wykres czasu w zależności od liczby wątków.}
    \label{wykres_czas}
\end{figure}

\begin{figure}[ht!]
    \centering
	\begin{tikzpicture}
		\begin{axis}[xlabel={Ilość wątków}, ylabel={Przyspieszenie}, legend style={legend pos=outer north east}]
		\addplot table [x=Wątki, y=Przyspieszenie, col sep=comma] {wyniki.csv};
		\end{axis}
	\end{tikzpicture}
    \caption{Wykres przyspieszenia w zależności od liczby wątków.}
    \label{wykres_przyspieszenie}
\end{figure}

\begin{figure}[ht!]
    \centering
	\begin{tikzpicture}
		\begin{axis}[xlabel={Ilość wątków}, ylabel={Wydajność}, legend style={legend pos=outer north east}]
		\addplot table [x=Wątki, y=Wydajność, col sep=comma] {wyniki.csv};
		\end{axis}
	\end{tikzpicture}
    \caption{Wykres wydajności w zależności od liczby wątków.}
    \label{wykres_wydajnosc}
\end{figure}

\end{document}
